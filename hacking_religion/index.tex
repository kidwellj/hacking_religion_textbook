% Options for packages loaded elsewhere
\PassOptionsToPackage{unicode}{hyperref}
\PassOptionsToPackage{hyphens}{url}
\PassOptionsToPackage{dvipsnames,svgnames,x11names}{xcolor}
%
\documentclass[
  letterpaper,
  DIV=11,
  numbers=noendperiod]{scrreprt}

\usepackage{amsmath,amssymb}
\usepackage{iftex}
\ifPDFTeX
  \usepackage[T1]{fontenc}
  \usepackage[utf8]{inputenc}
  \usepackage{textcomp} % provide euro and other symbols
\else % if luatex or xetex
  \usepackage{unicode-math}
  \defaultfontfeatures{Scale=MatchLowercase}
  \defaultfontfeatures[\rmfamily]{Ligatures=TeX,Scale=1}
\fi
\usepackage{lmodern}
\ifPDFTeX\else  
    % xetex/luatex font selection
\fi
% Use upquote if available, for straight quotes in verbatim environments
\IfFileExists{upquote.sty}{\usepackage{upquote}}{}
\IfFileExists{microtype.sty}{% use microtype if available
  \usepackage[]{microtype}
  \UseMicrotypeSet[protrusion]{basicmath} % disable protrusion for tt fonts
}{}
\makeatletter
\@ifundefined{KOMAClassName}{% if non-KOMA class
  \IfFileExists{parskip.sty}{%
    \usepackage{parskip}
  }{% else
    \setlength{\parindent}{0pt}
    \setlength{\parskip}{6pt plus 2pt minus 1pt}}
}{% if KOMA class
  \KOMAoptions{parskip=half}}
\makeatother
\usepackage{xcolor}
\setlength{\emergencystretch}{3em} % prevent overfull lines
\setcounter{secnumdepth}{5}
% Make \paragraph and \subparagraph free-standing
\ifx\paragraph\undefined\else
  \let\oldparagraph\paragraph
  \renewcommand{\paragraph}[1]{\oldparagraph{#1}\mbox{}}
\fi
\ifx\subparagraph\undefined\else
  \let\oldsubparagraph\subparagraph
  \renewcommand{\subparagraph}[1]{\oldsubparagraph{#1}\mbox{}}
\fi


\providecommand{\tightlist}{%
  \setlength{\itemsep}{0pt}\setlength{\parskip}{0pt}}\usepackage{longtable,booktabs,array}
\usepackage{calc} % for calculating minipage widths
% Correct order of tables after \paragraph or \subparagraph
\usepackage{etoolbox}
\makeatletter
\patchcmd\longtable{\par}{\if@noskipsec\mbox{}\fi\par}{}{}
\makeatother
% Allow footnotes in longtable head/foot
\IfFileExists{footnotehyper.sty}{\usepackage{footnotehyper}}{\usepackage{footnote}}
\makesavenoteenv{longtable}
\usepackage{graphicx}
\makeatletter
\def\maxwidth{\ifdim\Gin@nat@width>\linewidth\linewidth\else\Gin@nat@width\fi}
\def\maxheight{\ifdim\Gin@nat@height>\textheight\textheight\else\Gin@nat@height\fi}
\makeatother
% Scale images if necessary, so that they will not overflow the page
% margins by default, and it is still possible to overwrite the defaults
% using explicit options in \includegraphics[width, height, ...]{}
\setkeys{Gin}{width=\maxwidth,height=\maxheight,keepaspectratio}
% Set default figure placement to htbp
\makeatletter
\def\fps@figure{htbp}
\makeatother
\newlength{\cslhangindent}
\setlength{\cslhangindent}{1.5em}
\newlength{\csllabelwidth}
\setlength{\csllabelwidth}{3em}
\newlength{\cslentryspacingunit} % times entry-spacing
\setlength{\cslentryspacingunit}{\parskip}
\newenvironment{CSLReferences}[2] % #1 hanging-ident, #2 entry spacing
 {% don't indent paragraphs
  \setlength{\parindent}{0pt}
  % turn on hanging indent if param 1 is 1
  \ifodd #1
  \let\oldpar\par
  \def\par{\hangindent=\cslhangindent\oldpar}
  \fi
  % set entry spacing
  \setlength{\parskip}{#2\cslentryspacingunit}
 }%
 {}
\usepackage{calc}
\newcommand{\CSLBlock}[1]{#1\hfill\break}
\newcommand{\CSLLeftMargin}[1]{\parbox[t]{\csllabelwidth}{#1}}
\newcommand{\CSLRightInline}[1]{\parbox[t]{\linewidth - \csllabelwidth}{#1}\break}
\newcommand{\CSLIndent}[1]{\hspace{\cslhangindent}#1}

\KOMAoption{captions}{tableheading}
\makeatletter
\@ifpackageloaded{tcolorbox}{}{\usepackage[skins,breakable]{tcolorbox}}
\@ifpackageloaded{fontawesome5}{}{\usepackage{fontawesome5}}
\definecolor{quarto-callout-color}{HTML}{909090}
\definecolor{quarto-callout-note-color}{HTML}{0758E5}
\definecolor{quarto-callout-important-color}{HTML}{CC1914}
\definecolor{quarto-callout-warning-color}{HTML}{EB9113}
\definecolor{quarto-callout-tip-color}{HTML}{00A047}
\definecolor{quarto-callout-caution-color}{HTML}{FC5300}
\definecolor{quarto-callout-color-frame}{HTML}{acacac}
\definecolor{quarto-callout-note-color-frame}{HTML}{4582ec}
\definecolor{quarto-callout-important-color-frame}{HTML}{d9534f}
\definecolor{quarto-callout-warning-color-frame}{HTML}{f0ad4e}
\definecolor{quarto-callout-tip-color-frame}{HTML}{02b875}
\definecolor{quarto-callout-caution-color-frame}{HTML}{fd7e14}
\makeatother
\makeatletter
\makeatother
\makeatletter
\@ifpackageloaded{bookmark}{}{\usepackage{bookmark}}
\makeatother
\makeatletter
\@ifpackageloaded{caption}{}{\usepackage{caption}}
\AtBeginDocument{%
\ifdefined\contentsname
  \renewcommand*\contentsname{Table of contents}
\else
  \newcommand\contentsname{Table of contents}
\fi
\ifdefined\listfigurename
  \renewcommand*\listfigurename{List of Figures}
\else
  \newcommand\listfigurename{List of Figures}
\fi
\ifdefined\listtablename
  \renewcommand*\listtablename{List of Tables}
\else
  \newcommand\listtablename{List of Tables}
\fi
\ifdefined\figurename
  \renewcommand*\figurename{Figure}
\else
  \newcommand\figurename{Figure}
\fi
\ifdefined\tablename
  \renewcommand*\tablename{Table}
\else
  \newcommand\tablename{Table}
\fi
}
\@ifpackageloaded{float}{}{\usepackage{float}}
\floatstyle{ruled}
\@ifundefined{c@chapter}{\newfloat{codelisting}{h}{lop}}{\newfloat{codelisting}{h}{lop}[chapter]}
\floatname{codelisting}{Listing}
\newcommand*\listoflistings{\listof{codelisting}{List of Listings}}
\makeatother
\makeatletter
\@ifpackageloaded{caption}{}{\usepackage{caption}}
\@ifpackageloaded{subcaption}{}{\usepackage{subcaption}}
\makeatother
\makeatletter
\@ifpackageloaded{tcolorbox}{}{\usepackage[skins,breakable]{tcolorbox}}
\makeatother
\makeatletter
\@ifundefined{shadecolor}{\definecolor{shadecolor}{rgb}{.97, .97, .97}}
\makeatother
\makeatletter
\makeatother
\makeatletter
\makeatother
\ifLuaTeX
  \usepackage{selnolig}  % disable illegal ligatures
\fi
\IfFileExists{bookmark.sty}{\usepackage{bookmark}}{\usepackage{hyperref}}
\IfFileExists{xurl.sty}{\usepackage{xurl}}{} % add URL line breaks if available
\urlstyle{same} % disable monospaced font for URLs
\hypersetup{
  pdftitle={Hacking Religion: TRS \& Data Science in Action},
  pdfauthor={Jeremy H. Kidwell},
  colorlinks=true,
  linkcolor={blue},
  filecolor={Maroon},
  citecolor={Blue},
  urlcolor={Blue},
  pdfcreator={LaTeX via pandoc}}

\title{Hacking Religion: TRS \& Data Science in Action}
\author{Jeremy H. Kidwell}
\date{2023-09-29}

\begin{document}
\maketitle
\ifdefined\Shaded\renewenvironment{Shaded}{\begin{tcolorbox}[frame hidden, interior hidden, borderline west={3pt}{0pt}{shadecolor}, enhanced, boxrule=0pt, breakable, sharp corners]}{\end{tcolorbox}}\fi

\renewcommand*\contentsname{Table of contents}
{
\hypersetup{linkcolor=}
\setcounter{tocdepth}{2}
\tableofcontents
}
\bookmarksetup{startatroot}

\hypertarget{preface}{%
\chapter*{Preface}\label{preface}}
\addcontentsline{toc}{chapter}{Preface}

\markboth{Preface}{Preface}

This is a Quarto book.

To learn more about Quarto books visit
\url{https://quarto.org/docs/books}.

\bookmarksetup{startatroot}

\hypertarget{introduction-hacking-religion}{%
\chapter{Introduction: Hacking
Religion}\label{introduction-hacking-religion}}

\hypertarget{who-this-book-is-for}{%
\section{Who this book is for}\label{who-this-book-is-for}}

\hypertarget{why-this-book}{%
\section{Why this book?}\label{why-this-book}}

\hypertarget{the-hacker-way}{%
\section{The hacker way}\label{the-hacker-way}}

\begin{enumerate}
\def\labelenumi{\arabic{enumi}.}
\item
  Tell the truth
\item
  Do not deceive using beauty
\item
  Work transparently: research as open code using open data
\item
  Draw others in: produce reproducible research
\item
  Learn by doing
\end{enumerate}

\hypertarget{using-the-r-programming-language}{%
\section{Using the R programming
language}\label{using-the-r-programming-language}}

Why R?

Explain accelerated approach in this book, working from examples and
providing exposure to concepts in a streamlined way, pointing to other
resources

Point to other guides,

\hypertarget{other-useful-guides}{%
\section{Other useful guides:}\label{other-useful-guides}}

\href{https://r4ds.hadley.nz/}{R For Data Science 2e}
\href{https://melaniewalsh.github.io/Intro-Cultural-Analytics/welcome.html}{Intro
to Cultural Analytics and Python}
\href{https://datasciencebox.org/01-overview}{Data Science in a Box}

\bookmarksetup{startatroot}

\hypertarget{the-2021-uk-census}{%
\chapter{The 2021 UK Census}\label{the-2021-uk-census}}

\hypertarget{how-to-get-data}{%
\section{How to get data}\label{how-to-get-data}}

\hypertarget{what-is-data}{%
\section{What is data?}\label{what-is-data}}

\hypertarget{your-first-project-building-a-pie-chart}{%
\section{Your first project: building a pie
chart}\label{your-first-project-building-a-pie-chart}}

Importing data from a CSV file

Examining data:

\begin{verbatim}
dat
head
tail
\end{verbatim}

\begin{tcolorbox}[enhanced jigsaw, colback=white, opacityback=0, colframe=quarto-callout-tip-color-frame, toprule=.15mm, opacitybacktitle=0.6, bottomtitle=1mm, leftrule=.75mm, breakable, left=2mm, rightrule=.15mm, title=\textcolor{quarto-callout-tip-color}{\faLightbulb}\hspace{0.5em}{What is Religion?}, coltitle=black, arc=.35mm, titlerule=0mm, bottomrule=.15mm, toptitle=1mm, colbacktitle=quarto-callout-tip-color!10!white]

Content tbd

\end{tcolorbox}

\begin{tcolorbox}[enhanced jigsaw, colback=white, opacityback=0, colframe=quarto-callout-tip-color-frame, toprule=.15mm, opacitybacktitle=0.6, bottomtitle=1mm, leftrule=.75mm, breakable, left=2mm, rightrule=.15mm, title=\textcolor{quarto-callout-tip-color}{\faLightbulb}\hspace{0.5em}{Hybrid Religious Identity}, coltitle=black, arc=.35mm, titlerule=0mm, bottomrule=.15mm, toptitle=1mm, colbacktitle=quarto-callout-tip-color!10!white]

Content tbd

\end{tcolorbox}

\begin{tcolorbox}[enhanced jigsaw, colback=white, opacityback=0, colframe=quarto-callout-tip-color-frame, toprule=.15mm, opacitybacktitle=0.6, bottomtitle=1mm, leftrule=.75mm, breakable, left=2mm, rightrule=.15mm, title=\textcolor{quarto-callout-tip-color}{\faLightbulb}\hspace{0.5em}{What is Secularisation?}, coltitle=black, arc=.35mm, titlerule=0mm, bottomrule=.15mm, toptitle=1mm, colbacktitle=quarto-callout-tip-color!10!white]

Content tbd

\end{tcolorbox}

\bookmarksetup{startatroot}

\hypertarget{references}{%
\chapter*{References}\label{references}}
\addcontentsline{toc}{chapter}{References}

\markboth{References}{References}

\hypertarget{refs}{}
\begin{CSLReferences}{0}{0}
\end{CSLReferences}

\bookmarksetup{startatroot}

\hypertarget{survey-data-spotlight-project}{%
\chapter{Survey Data: Spotlight
Project}\label{survey-data-spotlight-project}}

\begin{tcolorbox}[enhanced jigsaw, colback=white, opacityback=0, colframe=quarto-callout-tip-color-frame, toprule=.15mm, opacitybacktitle=0.6, bottomtitle=1mm, leftrule=.75mm, breakable, left=2mm, rightrule=.15mm, title=\textcolor{quarto-callout-tip-color}{\faLightbulb}\hspace{0.5em}{How can we measure religion?}, coltitle=black, arc=.35mm, titlerule=0mm, bottomrule=.15mm, toptitle=1mm, colbacktitle=quarto-callout-tip-color!10!white]

Content tbd

\end{tcolorbox}

\bookmarksetup{startatroot}

\hypertarget{references-1}{%
\chapter*{References}\label{references-1}}
\addcontentsline{toc}{chapter}{References}

\markboth{References}{References}

\hypertarget{refs}{}
\begin{CSLReferences}{0}{0}
\end{CSLReferences}

\bookmarksetup{startatroot}

\hypertarget{mapping-churches-geospatial-data-science}{%
\chapter{Mapping churches: geospatial data
science}\label{mapping-churches-geospatial-data-science}}

\bookmarksetup{startatroot}

\hypertarget{references-2}{%
\chapter*{References}\label{references-2}}
\addcontentsline{toc}{chapter}{References}

\markboth{References}{References}

\hypertarget{refs}{}
\begin{CSLReferences}{0}{0}
\end{CSLReferences}

\bookmarksetup{startatroot}

\hypertarget{data-scraping-corpus-analysis-and-wordclouds}{%
\chapter{Data scraping, corpus analysis and
wordclouds}\label{data-scraping-corpus-analysis-and-wordclouds}}

\bookmarksetup{startatroot}

\hypertarget{references-3}{%
\chapter*{References}\label{references-3}}
\addcontentsline{toc}{chapter}{References}

\markboth{References}{References}

\hypertarget{refs}{}
\begin{CSLReferences}{0}{0}
\end{CSLReferences}

\bookmarksetup{startatroot}

\hypertarget{summary}{%
\chapter{Summary}\label{summary}}

An open textbook introducing data science to religious studies

\bookmarksetup{startatroot}

\hypertarget{references-4}{%
\chapter*{References}\label{references-4}}
\addcontentsline{toc}{chapter}{References}

\markboth{References}{References}

\hypertarget{refs}{}
\begin{CSLReferences}{0}{0}
\end{CSLReferences}



\end{document}
